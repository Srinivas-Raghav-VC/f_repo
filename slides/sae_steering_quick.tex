% !TEX TS-program = xelatex
\documentclass[aspectratio=169]{beamer}
\usetheme{metropolis}
\usepackage{graphicx}
\usepackage{amsmath, amssymb}
\usepackage{tikz}
\usepackage{standalone}
\usepackage{xcolor}
\usepackage{hyperref}
\hypersetup{colorlinks=true, urlcolor=blue}
\usepackage{iftex}
\ifXeTeX
  \setsansfont{Latin Modern Sans}
  \setmonofont{Latin Modern Mono}
\fi

\newcommand{\diagrampath}{../diagrams}

\title{Steering LLaMA-3.1 8B with SAEs:\\ A Simple, First-Principles Methodology}
\author{Srinivas Raghav V C\\ Supervisor: Prof. Krishnendu S. P.}
\date{\today}

\begin{document}
\maketitle

% 1 — Intuition & goals
\begin{frame}{Intuition & goals (Feynman-style)}
\small
\begin{itemize}
  \item \textbf{Goal}: Reduce \emph{Hindi semantics} while preserving English quality; not just block Devanagari script.
  \item \textbf{Mental model}: Early layers handle form; \textbf{mid layers} consolidate meaning; late layers lexicalize to tokens.
  \item \textbf{Move}: Add tiny \emph{feature valves} (SAEs) at a few mid layers to attenuate only the Hindi-meaning band; a \textbf{script-blind} guard schedules attenuation.
  \item \textbf{Proof}: Script-blind ES drops; PPL/KL steady; no redistribution/leakage; privacy near random.
\end{itemize}
\end{frame}

% 2 — Architecture & baselines
\begin{frame}{Where we edit + What we compare}
\begin{columns}[T,onlytextwidth]
\column{0.55\textwidth}
\centering \includestandalone[mode=tex,width=\linewidth]{\diagrampath/diagram_transformer_pipeline}
\column{0.45\textwidth}
\centering \includestandalone[mode=tex,width=\linewidth]{\diagrampath/diagram_lora_reft}
\end{columns}
\vspace{1mm}
\small LoRA edits \emph{weights}; ReFT edits \emph{representations}. Our SAE-gate intervenes directly in meaning space.
\end{frame}

% 3 — Mechanism: SAE-gate + controller
\begin{frame}{Mechanism: SAE-gate + Controller}
\begin{columns}[T,onlytextwidth]
\column{0.6\textwidth}
\centering \includestandalone[mode=tex,width=\linewidth]{\diagrampath/diagram_sae_gate}
\column{0.4\textwidth}
\centering \includestandalone[mode=tex,width=\linewidth]{\diagrampath/diagram_dynamic_gating}
\end{columns}
\vspace{1mm}
\small Encode $h\to z$, attenuate selected $z[\mathcal{I}]\leftarrow (1-\alpha)z[\mathcal{I}]$, decode and add the small residual correction. \emph{Semantic} gating schedules $\alpha$ without token penalties.
\end{frame}

% 4 — Methodology: choose layers, define controls
\begin{frame}{Methodology: choose layers + define controls}
\begin{columns}[T,onlytextwidth]
\column{0.55\textwidth}
\centering \includestandalone[mode=tex,width=\linewidth]{\diagrampath/diagram_layer_selection}
\column{0.45\textwidth}
\centering \includestandalone[mode=tex,width=\linewidth]{\diagrampath/diagram_script_scrub}
\end{columns}
\vspace{1mm}
\small Pick top-$k$ mid layers by CKA/Procrustes/ANC; linear script scrub (INLP/LEACE-lite) serves as a control baseline.
\end{frame}

% 5 — Evaluation: metrics -> gates -> decision
\begin{frame}{Evaluation: metrics \textrightarrow{} gates \textrightarrow{} decision}
\centering
\includestandalone[mode=tex,width=0.9\linewidth]{\diagrampath/diagram_metrics_gates}
\vspace{1mm}
\small Proceed only if ES (script-aware \& script-blind) drop, retain PPL/KL steady, and safety axes (probes, x-ling ES, MIA) pass.
\end{frame}

% 6 — Lessons & next steps
\begin{frame}{Lessons from the field (compute \& cost)}
\small
\begin{itemize}
  \item I attempted end-to-end runs on larger models; \textbf{compute/offload limits and VRAM spikes} (especially SAE sizes) led to repeated failures and unexpected cloud costs.
  \item I paused experiments and \textbf{returned to first principles}: clarified the architecture, defined falsifiable gates, added script-blind controllers, and designed visual diagnostics (these slides & diagrams).
  \item Next: when compute is available, run the dose–response sweep and confirm the gates on both synthetic and small real sets.
\end{itemize}
\end{frame}

\end{document}
